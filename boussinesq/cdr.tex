\documentclass[a4paper,12pt,oneside]{book}
% oneside here allows chapter starting from any page
\usepackage{color}
\usepackage[margin=2cm]{geometry}

% this package is needed for subplotting
\usepackage{graphicx}
\usepackage{subcaption}
\usepackage[font=small,labelfont=bf]{caption}

% amsmath is needed for write subequation.
\usepackage{amsmath}

\usepackage{mathtools}

% set the indentation of the paragraph
\setlength\parindent{0pt}
%\setlength{\parindent}{2em}

% set the spacing between paragraphs
\setlength{\parskip}{\baselineskip}

%% \usepackage{titling}
%% \setlength{\droptitle}{-3cm}

% make the chapter title centered
\usepackage{titlesec}
\titleformat{\chapter}[display]
{\normalfont\huge\bfseries}
{\centering\chaptertitlename\ \thechapter}{0pt}{\Large}
\titlespacing*{\chapter}{0pt}{-30pt}{0pt}   % to reduce the spacing
                                % for titles
\titlespacing\section{0pt}{12pt plus 4pt minus 2pt}{0pt plus 2pt minus 2pt}
\renewcommand\thesection{\Alph{section}}

% change chapter into career episode
%\renewcommand{\chaptername}{Career Episode}

\renewcommand\thesection{\Alph{section}}

% change chapter into career episode
\renewcommand{\chaptername}{CIVL3150 Note}


\newcounter{parnum}[chapter]
\renewcommand{\theparnum}{\thechapter.\arabic{parnum}}


%% this section creates a count1r for paragraphs, the format is like
%(chapter number, paragraph number)
\newcommand{\N}{%
  \refstepcounter{parnum}%
% with \makebox[\parindent][l], the paragraph will overlap from counter
%  \makebox[\parindent][l]{\par{(\theparnum)\space}}}
  {\par{[\theparnum]\space}}}
%  {\par{(\theparnum)\space}}} % this one is working



\begin{document}
\title{Derivation of groundwater flow equation}
\date{September 18, 2014}

%% \newcounter{paranum}
%% \newcommand{\P}{\vspace{10pt}\noindent\textbf{\refstepcounter{paranum}\theparanum}


%% \newcounter{paranum}[section]
%% \newcommand{\P}{\vspace{10pt}\noindent\textbf{\thesection.\refstepcounter{paranum}\theparanum}\textbf}

\chapter{Derivation of groundwater flow equation}


\N The Boussnesq's approximation of groundwater flow is a nonlinear parabolic partial differential equation. Under cartesian coordinate, it reads:

\begin{equation}
S_{y}\frac{\partial h}{\partial t}=
\frac{\partial }{\partial x}\left(kh \frac{\partial h}{\partial x}\right)+ 
\frac{\partial }{\partial y}\left(kh \frac{\partial h}{\partial y}\right)
\label{equ:control}
\end{equation}


%% \begin{equation}
%% S_{y}\frac{\partial h}{\partial t}=
%% \frac{\partial }{\partial x}\left(kh \frac{\partial h}{\partial x}\right)+ 
%% \frac{\partial }{\partial y}\left(kh \frac{\partial h}{\partial y}\right)
%% \label{equ:control}
%% \end{equation}


where $S_{y}$ is the specific yield, $x$ and $y$ are spatial coordinates, $t$ is time, $h$ is hydraulic head (assuming the datum is at the impermeable solid layer at the bottom of the aquifer), $k$ is hydraulic conductivity.

\N Now we introduce the polar coordinates $r$ and $\theta$, which have the relations:

\begin{equation}
r=\sqrt{x^{2}+y^{2}}
\end{equation}
\begin{equation}
\theta=\arctan \frac{y}{x}
\end{equation}
or equivalently:
\begin{equation}
x=r\cos\theta
\end{equation}
\begin{equation}
y=r\sin\theta
\end{equation}

\N These relationships result in the following derivatives:

\begin{equation}
\frac{\partial r}{\partial x}=\frac{1}{2} \frac{2x}{\sqrt{x^{2}+y^{2}}}
=\cos\theta
\label{equ:label1}
\end{equation}
\begin{equation}
\frac{\partial r}{\partial y}=
\sin\theta
\end{equation}
\begin{equation}
\frac{\partial\theta}{\partial x}=\frac{1}{1+\frac{y^{2}}{x^{2}}} 
\left(-\frac{y}{x^2}\right)=
\frac{-y}{x^{2}+y^{2}}=-\frac{\sin \theta}{r}
\end{equation}
\begin{equation}
\frac{\partial\theta}{\partial x}=\frac{1}{1+\frac{y^{2}}{x^{2}}} 
\left(\frac{1}{x}\right)=
\frac{x}{x^{2}+y^{2}}=\frac{\cos \theta}{r}
\label{equ:label2}
\end{equation}

\N According to chain rule, equation ~\ref{equ:control} can then be changed as:
\begin{equation}
\begin{split}
S_{y}\frac{\partial h}{\partial t}=&
\frac{\partial }{\partial r}      % first term
  \left[kh 
     \left(
       \frac{\partial h}{\partial r}
       \frac{\partial r}{\partial x}
       +
       \frac{\partial h}{\partial \theta}
       \frac{\partial \theta}{\partial x}
     \right)  
  \right]
\frac{\partial r}{\partial x}
+  
\frac{\partial }{\partial \theta}      % second term
   \left[kh 
     \left(
        \frac{\partial h}{\partial r}
        \frac{\partial r}{\partial x}
        +
        \frac{\partial h}{\partial \theta}
        \frac{\partial \theta}{\partial x}
     \right)
    \right]
\frac{\partial \theta}{\partial x}+ \\
&\frac{\partial }{\partial r}     % third therm
    \left[kh 
      \left(
        \frac{\partial h}{\partial r}
        \frac{\partial r}{\partial y}
        +
        \frac{\partial h}{\partial \theta}
        \frac{\partial \theta}{\partial y}
       \right)  
     \right]
\frac{\partial r}{\partial y}
+\frac{\partial }{\partial \theta}  % fourth term
    \left[kh 
      \left(
        \frac{\partial h}{\partial r}
        \frac{\partial r}{\partial y}
        + 
        \frac{\partial h}{\partial \theta}
        \frac{\partial \theta}{\partial y}
      \right)
    \right] 
\frac{\partial \theta}{\partial y}
\end{split}
\label{equ:control2}
\end{equation}

\N Note that $\frac{\partial h }{\partial \theta}$ is zero while  
$\frac{\partial }{\partial \theta}\left(\frac{\partial r}{\partial x}\right)$ 
is non-zero, equation ~\ref{equ:control2} simplifies to: 

\begin{equation}
\begin{split}
S_{y}\frac{\partial h}{\partial t}=&
\frac{\partial }{\partial r}   % first therm
  \left[kh 
     \left(
       \frac{\partial h}{\partial r}
       \frac{\partial r}{\partial x}
     \right)  
  \right]
\frac{\partial r}{\partial x}
+ 
kh                             % second term
\frac{\partial h}{\partial r} 
    \left[
    \frac{\partial}{\partial \theta}
        \left(
            \frac{\partial r}{ \partial x}
        \right)
    \right]
\frac{\partial \theta}{\partial x} 
+\\
&\frac{\partial }{\partial r}   % third term 
    \left[kh 
      \left(
        \frac{\partial h}{\partial r}
        \frac{\partial r}{\partial y}
      \right)
    \right] 
\frac{\partial r}{\partial y}
+
kh\frac{\partial h}{\partial r}   % fourth term
    \left[
        \frac{\partial }{\partial \theta}
            \left( 
            \frac{\partial r}{\partial y} 
            \right)
    \right]
\frac{\partial \theta}{\partial y}
\label{equ:control3}
\end{split}
\end{equation}

\N Critical notice: 
$\frac{\partial}{\partial \theta}\left(\frac{\partial r}{\partial x}\right)
\neq\frac{\partial}{\partial x}\left(\frac{\partial r}{\partial \theta}\right)
\neq\frac{\partial^{2} r}{\partial \theta \partial r}$ 
as $x$ and $\theta$ are depedent on each other!



\N Inserting equations ~\ref{equ:label1} to ~\ref{equ:label2} into equation ~\ref{equ:control3} gives:
\begin{equation}
\begin{split}
S_{y}\frac{\partial h}{\partial t}=&
\frac{\partial }{\partial r}   % first term
  \left[kh 
     \left(
       \frac{\partial h}{\partial r}
       \cos\theta
     \right)  
  \right]
\cos\theta
+  % second term
kh                             % second term
\frac{\partial h}{\partial r} 
    \left(
      -\sin\theta
    \right)
\left(
-\frac{\sin\theta}{r}
\right)
+ \\
&\frac{\partial }{\partial r}  % third therm
    \left[kh 
      \left(
        \frac{\partial h}{\partial r}
        \sin\theta
      \right)
    \right] 
\sin\theta
+
kh\frac{\partial h}{\partial r}   % fourth term
    \left(
       \cos\theta
    \right)
\left(
\frac{\cos\theta}{r}
\right)\\
&=\frac{\partial }{\partial r}   % first term
  \left[kh 
     \left(
       \frac{\partial h}{\partial r}
     \right)  
  \right]
+
\frac{kh}{r}\frac{\partial h}{\partial r}\\
&=\frac{1}{r}
\left\{
  r\frac{\partial }{\partial r}
  \left[kh 
     \left(
       \frac{\partial h}{\partial r}
     \right)  
  \right]
+
\frac{\partial r}{\partial r} kh\frac{\partial h}{\partial r}
\right\}\\
&= \frac{1}{r}\frac{\partial }{\partial r}
\left[rkh \frac{\partial h}{\partial r}
\right]
\label{equ:control5}
\end{split}
\end{equation}

\N For a steady state condition, $\frac{\partial h}{\partial t}=0$, equation ~\ref{equ:control5} can be solved using the separation of variables:

\begin{equation}
\frac{k}{2}h^{2}=c_{1}\ln r+c_{2}
\label{equ:solution2}
\end{equation}

where $c_{1}$ and $c_{2}$ are constants. Inputing the boundary 
condition to equation ~\ref{equ:solution2}, specifically:

\begin{equation}
  \text{boundary conditions}
  \begin{cases}
    h=h_{gw}, & r=r_{gw}.\\
    h=h_{vw}, & r=r_{vw}. 
  \end{cases}
\end{equation}

we obtain:

\begin{equation}
  \text{}
  \begin{cases}
    c_{1}=\frac{k\left(h_{vw}^{2}-h_{gw}^{2}\right)}{2\left(\ln r_{vw} -\ln r_{gw} \right)}\\
    c_{2}=\frac{k}{2}h_{gw}^{2}-
     \frac{k\left(h_{vw}^{2}-h_{gw}^{2}\right)}{2\left(\ln r_{vw} -\ln r_{gw} \right)}.& 
  \end{cases}
\end{equation}

\N In equation ~\ref{equ:solution2}, since $k$, $c_1$ and $c_2$ are constants, we can get:

\begin{equation}
  \frac{\partial h}{\partial r}=\frac{1}{2}\frac{1}{\sqrt{\frac{2}{k}(c_1\ln r+c_2)}}
  \frac{2}{k}c_1\frac{1}{r}
  =\frac{c_1}{hkr}
\end{equation}

\N Note that since the datum is assumed to be at the impermeable layer,
the cross section area of a circle with radius $r_c$ can be represented by
$2 \pi r_c h_c$, and hence the flow rate (positive outward at polar coordinate) through 
cylinder with radius $r_c$ and hydraulic head $h_c$ can be calculated by:

\begin{equation}
Q=-2\pi rkh \frac{\partial h}{\partial r}\vert_{r=r_c,h=h_c}=-2\pi rkh\frac{c_1}{hkr}  \vert_{r=r_c,h=h_c}=-\frac{\pi k\left(h_{vw}^{2}-h_{gw}^{2}\right)}{\left(\ln r_{vw} -\ln r_{gw} \right)}
\end{equation}

which is the flow equation in the project 2. The negative symbol is taken as it is positive inward in the project.



\end{document}
