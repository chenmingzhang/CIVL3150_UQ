\documentclass[a4paper,12pt,oneside]{book}
% oneside here allows chapter starting from any page
\usepackage{color}
\usepackage[margin=2cm]{geometry}

% this package is needed for subplotting
\usepackage{graphicx}
\usepackage{subcaption}
\usepackage[font=small,labelfont=bf]{caption}

% amsmath is needed for write subequation.
\usepackage{amsmath}

\usepackage{mathtools}

% set the indentation of the paragraph
\setlength\parindent{0pt}
%\setlength{\parindent}{2em}

% set the spacing between paragraphs
\setlength{\parskip}{\baselineskip}

%% \usepackage{titling}
%% \setlength{\droptitle}{-3cm}

% make the chapter title centered
\usepackage{titlesec}
\titleformat{\chapter}[display]
{\normalfont\huge\bfseries}
{\centering\chaptertitlename\ \thechapter}{0pt}{\Large}
\titlespacing*{\chapter}{0pt}{-30pt}{0pt}   % to reduce the spacing
                                % for titles
\titlespacing\section{0pt}{12pt plus 4pt minus 2pt}{0pt plus 2pt minus 2pt}
\renewcommand\thesection{\Alph{section}}

% change chapter into career episode
%\renewcommand{\chaptername}{Career Episode}

\renewcommand\thesection{\Alph{section}}

% change chapter into career episode
\renewcommand{\chaptername}{CIVL3150 Note}


\newcounter{parnum}[chapter]
\renewcommand{\theparnum}{\thechapter.\arabic{parnum}}


%% this section creates a count1r for paragraphs, the format is like
%(chapter number, paragraph number)
\newcommand{\N}{%
  \refstepcounter{parnum}%
% with \makebox[\parindent][l], the paragraph will overlap from counter
%  \makebox[\parindent][l]{\par{(\theparnum)\space}}}
  {\par{[\theparnum]\space}}}
%  {\par{(\theparnum)\space}}} % this one is working



\begin{document}
\title{Derivation of groundwater flow equation}
\date{September 18, 2014}

%% \newcounter{paranum}
%% \newcommand{\P}{\vspace{10pt}\noindent\textbf{\refstepcounter{paranum}\theparanum}


%% \newcounter{paranum}[section]
%% \newcommand{\P}{\vspace{10pt}\noindent\textbf{\thesection.\refstepcounter{paranum}\theparanum}\textbf}

\chapter{Derivation of groundwater flow equation}


\N The Boussnesq's approximation of groundwater flow is a nonlinear parabolic partial differential equation. Under cartesian coordinate, it reads:

\begin{equation}
S_{y}\frac{\partial h}{\partial t}=
\frac{\partial }{\partial x}\left(kh \frac{\partial h}{\partial x}\right)+ 
\frac{\partial }{\partial y}\left(kh \frac{\partial h}{\partial y}\right)
\label{equ:control}
\end{equation}


%% \begin{equation}
%% S_{y}\frac{\partial h}{\partial t}=
%% \frac{\partial }{\partial x}\left(kh \frac{\partial h}{\partial x}\right)+ 
%% \frac{\partial }{\partial y}\left(kh \frac{\partial h}{\partial y}\right)
%% \label{equ:control}
%% \end{equation}


where $S_{y}$ is the specific yield, $x$ and $y$ are spatial coordinates, $t$ is time, $h$ is hydraulic head, $k$ is hydraulic conductivity.

\N Now we introduce the polar coordinates $r$ and $\theta$, which have the relations:

\begin{equation}
r=\sqrt{x^{2}+y^{2}}
\end{equation}
\begin{equation}
\theta=\arctan \frac{y}{x}
\end{equation}
or equivalently:
\begin{equation}
x=r\cos\theta
\end{equation}
\begin{equation}
y=r\sin\theta
\end{equation}

\N These relationships result in the following derivatives:

\begin{equation}
\frac{\partial r}{\partial x}=\frac{1}{2} \frac{2x}{\sqrt{x^{2}+y^{2}}}
=\cos\theta
\label{equ:label1}
\end{equation}
\begin{equation}
\frac{\partial r}{\partial y}=
\sin\theta
\end{equation}
\begin{equation}
\frac{\partial\theta}{\partial x}=\frac{1}{1+\frac{y^{2}}{x^{2}}} 
\left(-\frac{y}{x^2}\right)=
\frac{-y}{x^{2}+y^{2}}=-\frac{\sin \theta}{r}
\end{equation}
\begin{equation}
\frac{\partial\theta}{\partial x}=\frac{1}{1+\frac{y^{2}}{x^{2}}} 
\left(\frac{1}{x}\right)=
\frac{x}{x^{2}+y^{2}}=\frac{\cos \theta}{r}
\label{equ:label2}
\end{equation}

\N According to chain rule, equation ~\ref{equ:control} can then be changed as:
\begin{equation}
\begin{split}
S_{y}\frac{\partial h}{\partial t}=&
\frac{\partial }{\partial r}      % first term
  \left[kh 
     \left(
       \frac{\partial h}{\partial r}
       \frac{\partial r}{\partial x}
       +
       \frac{\partial h}{\partial \theta}
       \frac{\partial \theta}{\partial x}
     \right)  
  \right]
\frac{\partial r}{\partial x}
+  
\frac{\partial }{\partial \theta}      % second term
   \left[kh 
     \left(
        \frac{\partial h}{\partial r}
        \frac{\partial r}{\partial x}
        +
        \frac{\partial h}{\partial \theta}
        \frac{\partial \theta}{\partial x}
     \right)
    \right]
\frac{\partial \theta}{\partial x}+ \\
&\frac{\partial }{\partial r}     % third therm
    \left[kh 
      \left(
        \frac{\partial h}{\partial r}
        \frac{\partial r}{\partial y}
        +
        \frac{\partial h}{\partial \theta}
        \frac{\partial \theta}{\partial y}
       \right)  
     \right]
\frac{\partial r}{\partial y}
+\frac{\partial }{\partial \theta}  % fourth term
    \left[kh 
      \left(
        \frac{\partial h}{\partial r}
        \frac{\partial r}{\partial y}
        + 
        \frac{\partial h}{\partial \theta}
        \frac{\partial \theta}{\partial y}
      \right)
    \right] 
\frac{\partial \theta}{\partial y}
\end{split}
\label{equ:control2}
\end{equation}

\N Note that $\frac{\partial h }{\partial \theta}$ is zero while  
$\frac{\partial }{\partial \theta}\left(\frac{\partial r}{\partial x}\right$ is non-zero,
 equation ~\ref{equ:control2} simplifies to: 

\begin{equation}
\begin{split}
S_{y}\frac{\partial h}{\partial t}=&
\frac{\partial }{\partial r}   % first therm
  \left[kh 
     \left(
       \frac{\partial h}{\partial r}
       \frac{\partial r}{\partial x}
     \right)  
  \right]
\frac{\partial r}{\partial x}
+ 
kh                             % second term
\frac{\partial h}{\partial r} 
    \left[
    \frac{\partial}{\partial \theta}
        \left(
            \frac{\partial r}{ \partial x}
        \right)
    \right]
\frac{\partial \theta}{\partial x} 
+\\
&\frac{\partial }{\partial r}   % third term 
    \left[kh 
      \left(
        \frac{\partial h}{\partial r}
        \frac{\partial r}{\partial y}
      \right)
    \right] 
\frac{\partial r}{\partial y}
+
kh\frac{\partial h}{\partial r}   % fourth term
    \left[
        \frac{\partial }{\partial \theta}
            \left( 
            \frac{\partial r}{\partial y} 
            \right)
    \right]
\frac{\partial \theta}{\partial y}
\label{equ:control3}
\end{splint}
\end{equation}

\N Critical notice: 
$\frac{\partial}{\partial \theta}\left(\frac{\partial r}{\partial x}\right)
\neq\frac{\partial}{\partial x}\left(\frac{\partial r}{\partial \theta}\right)
\neq\frac{\partial^{2} r}{\partial \theta \partial r}$ 
as $x$ and $\theta$ are depedent on each other!



\N Inserting equations ~\ref{equ:label1} to ~\ref{equ:label2} into equation ~\ref{equ:control3} gives:
\begin{equation}
\begin{split}
S_{y}\frac{\partial h}{\partial t}=&
\frac{\partial }{\partial r}   % first term
  \left[kh 
     \left(
       \frac{\partial h}{\partial r}
       \cos\theta
     \right)  
  \right]
\cos\theta
+  % second term
kh                             % second term
\frac{\partial h}{\partial r} 
    \left(
      -\sin\theta
    \right)
\left(
-\frac{\sin\theta}{r}
\right)
+ \\
&\frac{\partial }{\partial r}  % third therm
    \left[kh 
      \left(
        \frac{\partial h}{\partial r}
        \sin\theta
      \right)
    \right] 
\sin\theta
+
kh\frac{\partial h}{\partial r}   % fourth term
    \left(
       \cos\theta
    \right)
\left(
\frac{\cos\theta}{r}
\right)\\
&=\frac{\partial }{\partial r}   % first term
  \left[kh 
     \left(
       \frac{\partial h}{\partial r}
     \right)  
  \right]
+
\frac{kh}{r}\frac{\partial h}{\partial r}\\
&=\frac{1}{r}
\left\{
  r\frac{\partial }{\partial r}
  \left[kh 
     \left(
       \frac{\partial h}{\partial r}
     \right)  
  \right]
+
\frac{\partial r}{\partial r} kh\frac{\partial h}{\partial r}
\right\}\\
&= \frac{1}{r}\frac{\partial }{\partial r}
\left[rkh \frac{\partial h}{\partial r}
\right]
\label{equ:control5}
\end{split}
\end{equation}
\N For a steady state condition, $\frac{\partial h}{\partial t}=0$, equation ~\ref{equ:control5} can be solved using the separation of variables:
\begin{equation}
\frac{k}{2}h^{2}=c_{1}\ln r+c_{2}
\label{equ:solution2}
\end{equation}
where $c_{1}$ and $c_{2}$ are constants. Inputing the boundary condition in equation ~\ref{equ:solution}, specifically:
\begin{equation}
  \text{boundary conditions}
  \begin{cases}
    h=h_{gw}, & r=r_{gw}.\\
    h=h_{vw}, & r=r_{vw}.& 
  \end{cases}
\end{equation}
we obtain:

\begin{equation}
  \text{}
  \begin{cases}
    c_{1}=\frac{k\left(h_{vw}^{2}-h_{gw}^{2}\right)}{2\left(\ln r_{vw} -\ln r_{gw} \right)}\\
    c_{2}=\frac{k}{2}h_{gw}^{2}-
     \frac{k\left(h_{vw}^{2}-h_{gw}^{2}\right)}{2\left(\ln r_{vw} -\ln r_{gw} \right)}.& 
  \end{cases}
\end{equation}

\N In equation ~\ref{equ:solution2}, since $k$, $c_1$ and $c_2$ are constants, we can get:
\begin{equation}
\frac{\partial h}{\partial r}=\frac{1}{2}\frac{1}{\sqrt{\frac{2}{k}(c_1\ln r+c_2)}}
\frac{2}{k}c_1\frac{1}{r}
=\frac{c_1}{hkr}
\end{equation}

\N Note that the flow rate (positive outward at polar coordinate) through cylinder with radius $r_c$ and hydraulic head $h_c$ can be calculated by:
\begin{equation}
Q=-2\pi rkh \frac{\partial h}{\partial r}\vert_{r=r_c,h=h_c}=-2\pi rkh\frac{c_1}{hkr}  \vert_{r=r_c,h=h_c}=-\frac{\pi k\left(h_{vw}^{2}-h_{gw}^{2}\right)}{\left(\ln r_{vw} -\ln r_{gw} \right)}
\end{equation}

which is the flow equation in the project 2. The negative symbol is taken as it is positive inward in the project.



%% \section{INTRODUCTION}

%% \textbf{Duration}: January 2014 to Continuing

%% \textbf{Name of the Employer}: School of Civil Engineering, The
%% University of Queensland, Brisbane, Australia

%% \textbf{Designation}: Research Assistant

%% \section{BACKGROUND}

%% \N Urban network sewage network serves for carrying water from
%% individual household to the waste water treatment plant (WWTP).
%% The WWTP has its maximum capacity to process waste water up to the
%% quality which do not bring harm to the natural environment after
%% recycling.
%% However, with the possibility of storm water inflow to the sewer
%% network, and the occasionally large amount of waste water being injected
%% into the sewer system, the WWTP may be subject to process large quantity
%% of waste water that is beyond its capacity.
%% Therefore, it is essential to establish a waste water forecasting system
%% that is able to predict the incoming volume of waste water to the WWTP
%% so that necessary decisions and actions can be arranged before the large
%% amount of waste water coming into the WWTP.
%% In addition, if substantial amount of contaminants have been
%% intentionally or accidentally dumped into the sewer network, the
%% normal procedures to clean the waste water may not be able to process
%% the water to the required quality.
%% Therefore, it is required to estimate the elapsed time for the
%% contaminant travelling through the sewer network into WWTP so that
%% proper reaction can be arranged properly.


%% \N There are two objectives within the project: 
%% the first one is to develop a series of sensors to acquire in situ
%% real-time and continuously critically important sewage quality data. 
%% This objective is mainly conducted by research staffs at the Griffith
%% University.
%% The second objectives is to develop a process based model that is able
%% to estimate the waste water flow rate and travel time of the
%% contaminant within the sewer network.
%% Being my major task in this project, the second objective is mainly
%% conducted by scientist and engineers at the university of Queensland. 

%% \N This project is co-funded by Australian Research Council linkage
%% project, as well as industrial partners, including Sydney Water,
%% Melbourne Water, Water Corporation of Western Australia, Gold Coast
%% water.

%% \begin{figure}[!htb]
%%   \centering
%%   \includegraphics[scale=1]{Figure1.eps}
%%   \caption{Organization structure of the Sewer network model project}
%%   \label{fig:digraph}
%% \end{figure}

%% \section{PERSONAL ENGINEERING ACTIVITY}

%% \N My total work has been divided into three scenarios: 
%% \begin{enumerate}
%%   \item evaluating the currently existing models. 
%%   \item developing a flow and mass transport model in the sewer
%%     network.
%%   \item Validating the developed model to the sewer network that is
%%     monitored in our project.
%% \end{enumerate}

%% \N Within the first three months, I have conducted thorough literature
%% review, and made extensive communications with both industrial
%% partners and the other research group. 
%% By this method, I found the current existing models (e.g., SWMM,
%% MOUSE) rely heavily on the accurate prescription of boundary
%% conditions in the networks, such as flow rate from the pressure pump,
%% rainfall rate. 
%% While the direct measurement of water depth and water quality in the
%% junctions are just used for validating the model performance.
%% Therefore, it is essential to establish a model, that is able to
%% directly use the measurements in the junctions to predict the mass and
%% transport. 

%% \N The flow and mass transport model I have developed aims to
%% calculate the flow condition and mass transport time at one sewer path
%% where flow and mass gets transported. There are four steps in using
%% this model:
%% \begin{enumerate}
%%   \item Delineate the flow transport path within the sewer network.
%%   \item Inspect the available measurement data within the system.
%%   \item Use the height measurement data to calculate the flow rate.
%%   \item Predict the mass transport time within the flow transport path.
%% \end{enumerate}

%% \N I found there are two characteristics that general sewer networks
%% follows: the first characteristic is that sewer networks are similar
%% to the tree structures.
%% The waste water source, such as residential and industrial
%% catchments are located at the branches, while the WWTP is located in
%% the main path.
%% Another characteristic of the general sewer networks is
%% that flow in the sewer network is mainly driven by gravity. 
%% In such configuration, the wastewater source are located at the higher
%% altitude, while the WWTP is located in the lower altitude. 
%% According to the characteristics of general sewer networks, I have
%% made two guidelines to delineate the path:
%% The first guideline is that the next junction where the
%% selected conduit leads to should be closer to the WWTP than the
%% previous junction. 
%% The second guideline is that flow preferentially
%% goes through the conduit with steeper gravity gradient ($S_o$). 
%% Figure ~\ref{fig:structure} shows an example of the sewer network and
%% two flow path.


%% \begin{figure}
%%  \vspace*{\fill}
%%   \begin{subfigure}[b]{1\textwidth}
%%     \centering
%%     \includegraphics[width=8cm]{Delineate.jpg}
%%     \caption{}
%%     \label{fig:test1}
%%     \vspace*{\fill}
%%   \end{subfigure}
%%   \begin{subfigure}[b]{1\textwidth}
%%     \centering
%%     \includegraphics[width=15cm]{path1.jpg}
%%     \caption{}
%%     \label{fig:test2}
%%   \end{subfigure}
%%   \begin{subfigure}[b]{1\textwidth}
%%     \centering
%%     \includegraphics[width=15cm]{path2.jpg}
%%     \caption{}
%%     \label{fig:test3}
%%   \end{subfigure}
%%   \caption{Sewer network and two flow transport paths. The black the
%%     yellow line lines are conduits, the black dots are junctions. The
%%     red and blue lines are the flow paths from injecting point A and
%%     B, respectively, to the WWTP. The geometry of the two paths are
%%     displayed in (b) and (c)}
%%   \label{fig:structure}
%% \end{figure}



%% %% The second step is to find the measurement data available throughout
%% %% the path. 
%% %% generally the water depth is monitored at the junctions. 
%% %% In the case displayed in figure 2(a), we assume that all the water
%% %% depth is constantly measured at each junctions of the path. 
%% %% The third step involves using the measurement data to calculate the flow rate through each conduit of the

%% %% It is worthy to note that some subcatchment can be located in the lower altitude using pressure station to pump the wastewater to the high elevation. The flow path through the pressure stations are not considered in the model. This function can be added later in the model. 

%% %% obtaining the flow condition and mass transport time in the sewer network requires the flow rate through one path where mass gets transported. This requires (1) the geometry of the conduits, (2) the 

%% \N The model I developed is based on momentum balance equation with
%% diffusive water approximation:

%% \begin{equation}
%% Q_{Conduit}=\frac{r^{8/3}S^{1/2}}{2^{5/3}n} 
%%   \frac{(\theta_{junction}-\sin
%%     \theta_{junction})^{5/3}}{\theta^{2/3}}
%% \label{fig:conduitequation}
%% \end{equation}

%% \N where $Q_{conduit}$ is the cliometric flow in the conduit $r$ is the
%% radius of the conduit, $\theta_{junction}$ is the central angle of the
%% wetting area which is directly linked to the water depth in the
%% conduit and junction (Figure ~\ref{overflow}), $S_o$ is the
%% gravitational slope,of the conduit, $n$ is Manning coefficient. By
%% equation ~\ref{fig:conduitequation}, the flow rate through one conduit
%% can be obtained directly by measuring the height at its upstream
%% junction. Figure ~\ref{fig:flowrate} displays the flow rate over time
%% at each conduit within the path 1 using different methods,
%% including the method I developed in the project and SWMM. Good
%% agreement between the two models, suggesting that directly using the
%% height at the conduit can still better estimate the flow rate in the
%% path with less computational effort. 

%% \begin{figure}[!htb]
%% \centering
%% \includegraphics[width=60mm]{Geometry.jpg}
%% \caption{The cross-section view of the conduit}
%% \label{overflow}
%% \end{figure}

%% \begin{figure}[!htb]
%% \centering
%% \includegraphics[width=160mm]{flow.eps}
%% \caption{The comparison of flow rate over time at each conduit within
%%   path 1 using my model (red dot) and SWMM (solid line)}
%% \label{fig:flowrate}
%% \end{figure}

%% \N Once flow rate thorough the whole path is know, I can calculate the
%% mass transport time through the path $T_{total}$ using the following equation:

%% \begin{equation}
%%   T_{total}=\sum\limits_{i=1}^{N_{conduits}}\frac{l_i}{v_i}
%% \end{equation}

%% \N where $N_{conduits}$ is the number of conduits in the path, $l_i$ is
%% the length of the path, $v_i$ is the flow velocity of the path. With
%% the flow rate well predicted, the mass travel time calculated by the
%% model I developed also agrees well with the that obtained from SWMM,
%% as shown in table ~\ref{tab:transport}.

%% \begin{table}[h]
%%   \caption{Comparison of calculated transport time between the model I
%%     developed and SWMM}
%%   \scalebox{0.75}{%
%%    \begin{tabular}{l*{6}{c}r}
%%      Paths   &Injecting time [min]& Travel time(SWMM) [min]& Travel time
%%      (Model)[min]& absolute error[min] & relative error [\%]\\
%%      \hline
%%      path1   & 60 & 100 & 104 & 4 &  3.8  \\
%%      path2   & 120& 155 & 160 & 5 &  3.1  \\
%%    \end{tabular}
%%   }
%%   \label{tab:transport}
%% \end{table}

%% \N With the successful cooperation with research team in Griffith
%% University, who focuses their effort on developing sensors that can
%% continuously measure the flow depth and water quality at the key
%% points, an integrated model has been established. I have delivered
%% this model to the industrial partner at a workshop organized in Gold
%% Coast from 9Th to 10th, May. The industrial partners has positive
%% view on this method, and are willing to provide more data to further
%% improve the model, including add pressure stations, weirs and storage
%% tanks. Finally this model will be widely adopted in the WWTP for
%% estimating the incoming flow and mass from the sewer network.


%% \section{SUMMARY}
%% \N the summary of the project is given as follows:

%% \begin {enumerate}
%%   \item In this project, I have contributed to the development of the
%%     flow and mass transport model for the sewer network. This model
%%     has been incorporated with the sensor developed in the project by
%%     other research group to estimate the flow rate and transport time
%%     of mass in the sewer network. As the project continues, the model
%%     will be deployed on site to simulate the concurrent flow.
%%   \item My skills on the fluid mechanics and mathematics have
%%     significantly assist the development of the model. With these
%%     knowledge, the model can be developed just within 2 month.
%%   \item  This
%%     project have broaden my views about how scientific knowledge can
%%     be effectively adopted in the engineering applications. From this
%%     opportunity, I have established connections with the industrial
%%     partners, which may provide chance for further collaborations. 
%% \end {enumerate}



%% \chapter{}

%% \section{INTRODUCTION}

%% \textbf{Duration}: January 2010 to Jan 2014

%% \textbf{Name of the Employer}:School of Civil Engineering, The
%% University of Queensland, Brisbane, Australia

%% \textbf{Designation}: PhD research student

%% \section{BACKGROUND}
%% \N Evaporation is a key process in the water and energy cycle on the
%% earth’s surface. Accurate prediction of the evaporation rate with
%% better understanding of the underlying processes provides fundamental
%% information for environmental, agricultural and industrial
%% applications. 

%% \N This project aims to better understanding how evaporation rate
%% would be changed due to various soil conditions, e.g., different
%% liquid water saturation, salinity, and soil types.

%% \N This project is co-funded by Australia Research Council and
%% National Centre for Groundwater Research and Training.

%% \begin{figure}[!htb]
%% \centering
%% \includegraphics[scale=1]{Figure2.eps}
%% \caption{Organization structure of the evaporation project}
%% \label{fig:digraph2}
%% \end{figure}

%% \section{PERSONAL ENGINEERING ACTIVITY}

%% \N After comprehensive literature reviews, I found there are three
%% fundamental questions regarding on quantifying the evaporation rate:
%% \begin{enumerate}
%%   \item How should the transport of liquid water and vapor at the
%%     soil-air interface (surface resistance) be described physically
%%     during evaporation?
%%   \item How can an integrated model be developed to properly simulate
%%     the evaporation process under non-isothermal conditions?
%%   \item How does evaporation-induced salt precipitation in turn affect
%%     the evaporation processes?
%% \end{enumerate}

%% \N To answer these questions, I have separated the work flow into
%% four parts:
%% \begin{enumerate}
%%   \item Conducting laboratory equation to obtain evaporation rate at
%%     different liquid water saturation.
%%   \item Deriving an analytical model to describe the water transport
%%     through the soil-air interface.
%%   \item Developing a numerical to simulate the transport of water,
%%     solute and heat in soils
%%   \item Use the numerical model to simulate how solid salt will in turn
%%     affect the evaporation process
%% \end{enumerate}

%% \N To ensure better design of the the laboratory experiment, I have
%% conducted one-year literature review, understanding how previous
%% experiments were conducted. 
%% After the finalization of the experimental
%% design, I spent three month contacting companies in Australia and
%% overseas for purchasing cost-effective equipment, including moisture
%% sensors, tensiometers, relative humidity sensors, electrical scales,
%% soils, dataloggers, columns,tubing, infrared lamps and so on. 
%% At the same time, I have been communicating with technician for
%% constructing the soil column, and conducting risk assessments to obtain
%% the permission to work in the laboratory. 
%% Before assembling the equipment, I also have done the calibration of the
%% sensors to ensure the accuracy of the measurements
%% After all the equipment and permissions are ready, I spent one month in
%% completing the experimental setup. 
%% The schematic diagram of experimental setup is shown in Figure
%% ~\ref{fig:Experimentsetup}.

%% \N The experiment was conducted in an environmentally controlled
%% laboratory with temperature equal to $20\,^{\circ}\mathrm{C}$ and
%% relative humidity 50\%.  Gravels were firstly packed over a 5 cm
%% thickness at the bottom of the column in an attempt to enhance the
%% hydraulic connectivity between the bottom of the testing sand and
%% reservoir.
%% Sensors were then installed from the side of the column prior to sand
%% packing. 
%% Sand packing was done by iteratively pouring 2 cm thick dry sand and
%% then tapping the column wall to achieve uniform bulk density across the
%% column while ensuring the integrity of sensor networks. 

%% \N After switching on the valve between the column and water reservoir,
%% water gradually filled the column from the bottom up above the soil
%% surface. Through this method, the problem with entrapped air bubbles in
%% the pore space can be avoided or minimised. The column was then placed
%% still for 1 day to reach an equilibrium condition.
%% The experiment started with the valve to the reservoir closed and the
%% infrared lamp switched on. 
%% The experiment continued until the evaporation rate became very low. The
%% experiment was then repeated with different heat intensities from the
%% infrared lamp and soil types. 

%% \begin{figure}[!htb]
%% %\centering
%% \includegraphics[scale=1]{1DEXPERIMENTSETTINGS_15-july.eps}
%% \caption{Schematic views of (a) setup of the soil column experiment
%%   and (b) enlarged view of the soil-air interface. Note that the
%%   sensors in (b) only show the elevations at the centreline of the
%%   sensors. The figure is not drawn to scale.}
%% \label{fig:Experimentsetup}
%% \end{figure}
 
%% \N With the on going laboratory experiment, I spent one year in the
%% development of the analytical model that describe the water transport
%% through the soil-air interface. This is essential for the evaporation
%% prediction as the current-existing method were all based on one
%% specific soil, suggesting that they may not be able to be applied in
%% another soil type. The derived equation is written in the form of
%% surface resistance and expressed as follows:


%% \begin{equation}
%% K_{vr}^c=_2F_1\left[1;\lambda;1+\lambda;-\frac{\zeta}{2\delta\Psi_m}
%% \left(\frac{\Psi^{\lambda(1+n)}_m}{\theta_p\Psi^{\lambda(1+n)}_b}
%% -\sqrt{\frac{\Psi^{\lambda(1+n)}_m}{\theta_p\Psi^{\lambda(1+n)}_b}}\right)\right]
%% \label{equ:hyper}
%% \end{equation}

%% \N where $_2F_1$ is the hypergeometric function; $\zeta$ is a constant;
%% $\Psi_m$ and $\Psi_b$ are actual matric potential and air entry matric
%% potential of the soil, respectively; $\delta$ is the thickness of the
%% air diffusive layer above the surface; $\theta_p$ is porosity; $n$ is a
%% constant; $\lambda$ is pore size factor. 
%% Equation ~\ref{equ:hyper} describes the vapor conductivity through the
%% soil air-interface with different liquid water saturation in the
%% soil. 
%% I used Mathmatica and Sagemath to derive the two equations.
%% Once the liquid water saturation and the aerodynamic condition is
%% known, this equation can be used to calculate the actual evaporation
%% rate.

%% \begin{figure}[!htb]
%% \centering
%% \includegraphics[scale=0.6]{Fig9_ETvsTime.eps}
%% \caption{Evaporation rate versus time: results obtained from experiments
%%   (symbols) and model (lines). The time interval between two
%%   neighbouring data points is 6.67 hours}
%% \label{fig:ETvsTime}
%% \end{figure}

%% \N Figure ~\ref{fig:ETvsTime} compares the measured evaporation rate from
%% the experiment and the calculated evaporation rate by the newly
%% developed model. Clearly the model can properly estimate the evaporation
%% rate under different liquid water saturation in soils. 


%% \N After the analytical model being verified, I spent another two years
%% constructing the numerical model that is able to predict the transport
%% of mass, solute and heat in soils. Constructing this model is necessary
%% as the current-existing model failed to well describe how water gets
%% transported in the soil-air interface and subsequently how mass, solute
%% and heat distributed over temporal and spatial scales. The model is
%% based on the balance of mass, solute and heat in the soils:

%% \begin{subequations}
%%   \begin{equation}
%%    \frac{\partial \rho_l\theta_l}{\partial t}
%%      =-\frac{\partial \rho_l\theta_l}{\partial z}
%%      -\frac{\partial \rho_v\theta_v}{\partial z}
%%   \end{equation}    
%%   \begin{equation}
%%    \frac{\partial \rho_l \theta_l C}{\partial C}
%%       +\frac{\partial M} {\partial t}
%%      =-\frac{\partial \rho_l q_l C}{\partial z}
%%      +\frac{\partial}{\partial z}
%%       \left[\rho_l \theta_l (D_i + D_m) 
%%       \frac{\partial C}{\partial z}\right]
%%   \end{equation}
%%   \begin{equation}
%%   \begin{split}
%%    \frac{\partial}{\partial t}\left[\rho_sc_s(T-T_0)
%%      +\rho_lc_l\theta_l(T-T_0)\right]&=
%%       \frac{\partial}{\partial z}
%%       \left(\lambda_T\frac{\partial T}{\partial z}\right) \\
%%      &-\frac{\partial}{\partial z}
%%       \left[ L_0 \rho_l q_v + \rho_l c_p (T-T_0) q_v   \right]\\
%%      &-\frac{\partial}{\partial z}
%%       \left[ \rho_l c_l (T-T_0) q_l \right]
%%   \end{split}
%%   \end{equation}
%% \end{subequations}

%% \N where $\rho$ is density, $q$ is flux, $c$ is heat capacity, the
%% subscript $l$, $v$ and $s$ represent liquid, vapor and solid phase,
%% respectively. T is temperature, t is time, z is vertical direction
%% positive upward, C is solute concentration, $D_i$ and $D_m$ are
%% respectively molecular and hydrodynamic dispersion.
%% \N To develop a numerical model based on this three equation, I have
%% discretized and differentialized the equations using finite element
%% method, and written the code using FORTRAN. The pre- and post- pocessing
%% were conducted using PYTHON. To accelerate the computational process,
%% I have used high performance computers Barrine at the University of
%% Queensland to conduct the simulation.

%% \begin{figure}[!htb]
%% \centering
%% \includegraphics[scale=0.4]{M14.eps}
%% \caption{Comparison between experimental and model results for case M14
%%   , including (a) liquid water saturation, (b) temperature
%%   (c) evaporation rate and cumulative evaporation. The numbers in the
%%   legend of (a) and (b) represent the elevations of the sensors that
%%   acquire the data (positive upward). The datum was set at soil
%%   surface.}
%% \label{fig:M14}
%% \end{figure}


%% \N Figure ~\ref{fig:M14} compares the results of the numerical model and
%% that of the laboratory experiment. A good agreement between the
%% simulated and measured result, including the liquid water saturation at
%% two different soil depth, temperature at three soil depth and
%% evaporation rate.  The simulated results of the other soils and heat
%% intensities also show promising agreements, suggesting the numerical
%% model and derived analytical model can better describe the drying
%% process of the soils.


%% \begin{figure}[!htb]
%% \centering
%% \includegraphics[scale=0.6]{salt.eps}
%% \caption{Predicted and measured results for case SS: (a) liquid water
%%   saturation, (b) solute concentration, (c) temperature, (d) liquid
%%   water flux, (e) water vapor flux across the soil column, (f) porosity,
%%   (g) temporal change of drying front location in the NSL and (h)
%%   temporal changes of evaporation rate and solid salt depth on the soil
%%   surface. The evaporation stages based on the predicted evaporation
%%   rate are indicated on the upper part of (h)}
%% \label{fig:salt}
%% \end{figure}


%% \N My role on this project also involves on how the precipitated salt
%% would in turn affect the evaporation rate. Based on the laboratory
%% experiment and numerical modeling experience, I found that usually the
%% locations where solute precipitated in solid form during evaporation has
%% very low liquid water saturation, suggesting nearly no flow took
%% place. Therefore, I made an assumption that during evaporation, the
%% precipitated salt has no effect to the groundwater flow. This assumption
%% has been incorporated into my model and later the model were used to
%% simulate four evaporation cases where soil water is saline. It turns out
%% that the model can we estimate the change of temperature, liquid water
%% saturation, concentration and evaporation over time during the
%% soil-drying processes (Figure ~\ref{fig:salt}).


%% \N During the project, I have been
%% constantly communicating with my supervision team about the novelty of
%% the ideas, the methodology to solve the problem and the approach applied
%% to analyze the outcomes. With four-year working on this project, I have
%% established a good relationship with my supervision team. 
%% Apart from the
%% weekly report to the supervision team, I also regularly communicate with
%% my review panels by presentations and reports, updating the progress of
%% the project. 

%% \N In addition, I regularly attend the annual meeting organized by the
%% external funding body, National Centre for Groundwater Research and
%% Training (NCGRT).  
%% In the meetings, I met with many other
%% hydrogeologist in Australia. We exchange our ideas and progresses by
%% reports, posts and presentations.

%% \N Furthermore, I have attended several international conferences
%% (listed in CV), presenting our outcomes to the engineers and scientists
%% from over the world. 


%% \section{SUMMARY}
%% \N The summary of the this career episode is given as follows:
%% \begin{enumerate}
%%   \item Through this project, we have better understandings on how
%%     water gets transported through the soil-air interface, this
%%     understanding has significantly improved the estimation of
%%     evaporation rate from unsaturated soils.
%%   \item In the project, I have conducted the laboratory experiment,
%%     analytical model derivation and the numerical development, which
%%     are the core components of the project.
%%   \item My knowledge of phiscs, mathematics and computer science has
%%     significantly assisted many aspect the project, including the
%%     model development, experimental design and data analysis. Finally
%%     the project can be finished within the given time frame.
%% \end{enumerate}





%% %-----------------------------------------------------------------------

%% \chapter{}

%% \section{INTRODUCTION}

%% \textbf{Duration}: January 2014 to May 2014


%% \textbf{Name of the Employer}:School of Civil Engineering, The
%% University of Queensland, Brisbane, Australia


%% \textbf{Designation}: Teaching Assistant


%% \section{BACKGROUND}

%% \N Understanding the interaction between surface water and groundwater
%% is important to evaluate the water balance in the earth surface. The
%% University of Queensland provides an fourth-year undergranduate course
%% surface water and groundwater modelling to
%% deliver the principles underpin the interaction of surface water and
%% groundwater.

%% \N This course involves 7 teaching hours
%% every weeks for 14 weeks. In the end students are requied to submit a
%% project report as well as attending examinations. My role in the
%% project is to guide the student finish a project that involves
%% simulating the surface water flow in a river that is subjected to strong rain
%% fall and the groundwater flow in the riverbank. I have been employed
%% in this course as teaching assistance for 5 semesters. The average
%% student number in each semesters are about 40.

%% \section{PERSONAL ENGINEERING ACTIVITY}

%% \begin{figure}[!htb]
%% \centering
%% \includegraphics[scale=1]{Figure3.eps}
%% \caption{Organization structure of the undergraduate course}
%% \label{fig:Figure3}
%% \end{figure}


%% \N To be competent to be a good teaching assistance requires fully
%% understanding of the principles that underpins the hydrological and
%% hydrodynamic processes. I have regularly attended all the classes
%% taught by the lectures, understanding the mechanisms related to the
%% subject. In addition, I spent time in finishing the report
%% independly before the commencement of the project, making sure that I
%% have completely understood the projects. 

%% \N This project involves
%% simulating surface water flow and groundwater flow using different
%% models. The surface water model are based on St Venant equation:

%% \begin{equation}
%%  \frac{\partial Q}{\partial x}+\frac{\partial A}{\partial t}=0
%%  \end{equation}

%% \begin{equation}
%% \frac{\partial Q}{\partial t}
%% +\frac{\partial (Q^2/A)}{\partial x}
%% +gA\frac{\partial H}{\partial x}
%% +gA(S_f-S_o)=0
%% \end{equation}

%% \N By solving the two equations, the flow rate and cross-section along
%% the river can be obtained. The course also contains using different
%% numerical method to solve the equations, including lax-fredrich
%% method, leap-frog method and Mac-cormac method. I have adoped these
%% three methods to slove the three equations using MATLAB and conducted
%% the comparision about the difference among these models.

%% \N The course also covers the change of the river geometry. I have
%% developed the model that uses three different kinds of geometries:
%% rectangular, tripzoid, trapezoid with riparian zones. Figure
%% ~\ref{fig:trapezoid} shows the cross-sectional profile of the river
%% with the presence of the riparian zone.

%% \begin{figure}[!htb]
%% \centering
%% \includegraphics[width=150mm]{cross_section.jpg}
%% \caption{The cross-section view of the trapezoid river with riparian zone}
%% \label{fig:trapezoid}
%% \end{figure}


%% \N The model is applied to simulate the flow rate and water depth in a
%% river that is subjected to strong inflow from the upstream due to
%% heavy rainfall. The simulated result is shown in Figure
%% ~\ref{fig:flow_height}. The model is found to be able to
%% simulate the effect of the riparian zone, that is, there is a plateau
%% on the simulated water depth over time, which is induced by water
%% inflowing into the riparian zone as storage tank.

%% %% \N To allow the model to simulate the transport of flow in a river, I
%% %% have discretized and  differentialized the fundamental equation using
%% %% Matlab. 
%% %To well prepare ~\ref{fig:Figure3}

%% \begin{figure}[!htb]
%% \centering
%% \includegraphics[width=150mm]{Flow_height.jpg}
%% \caption{The flow rate and height at 0m, 1000m and 2000m  }
%% \label{fig:flow_height}
%% \end{figure}

%% \N The water depth was then incorporated into the groundwater model
%% PMWIN to simulate the groundwater flow and solute transport at the
%% riverbank perpendicular to the flow direction ( Figure
%% ~\ref{fig:groundwater} ). When the water level rises, the surface
%% water starts to infiltrate into the soil. While the water level
%% recedes, groundwater starts to exfiltrate back into the river.

%% \begin{figure}
%%  \vspace*{\fill}
%%   \begin{subfigure}[b]{1\textwidth}
%%     \centering
%%     \includegraphics[width=8cm]{5dat.jpg}
%%     \caption{}
%%     \label{fig:5dat}
%%     \vspace*{\fill}
%%   \end{subfigure}

%%   \begin{subfigure}[b]{1\textwidth}
%%     \centering
%%     \includegraphics[width=8cm]{9dat.jpg}
%%     \caption{}
%%     \label{fig:9dat}
%%   \end{subfigure}

%%   \caption{Simulated groundwater flow and hydraulic head in the
%%     riverbank using PMWIN}
%%   \label{fig:groundwater}
%% \end{figure}

%% \N This course requires me to better communicate with both lectures
%% and students. Generally the lectures focuses more on the principles of
%% the fluid dynamics and mathematical approaches, while less so on the
%% applications of these principles.
%% To ensure students can apply this
%% principles in the computing code and other scientific software, I
%% organized two-hour lecture introducing how to use MATLAB. At each
%% practical session, I always spend one hour teaching students how to
%% implement the fundamental equation into computer codes. After the
%% introduction session, I regularly check the progress of each student
%% group and the problem they have during the project ensuring every
%% group understands the course well and is able to finish the project
%% within the given time frame. 

%% \N After 14 week working on the project, students are required to
%% submit their report. I regularly mark these reports. The reports are
%% usually marked by two rules: 1. the quality of the report, 2. their
%% performance during at the lab sessions. Once the report are marked, I
%% submitted the result to the lecture and exchange our ideas about the
%% result. 

%% \section{SUMMARY}

%% \N The summary for this episode is given as follows:
%% \begin{enumerate}
%%   \item Through this episode, I have gained better understandings about how
%% to use numerical tools to simulate the interaction of surface water
%% and groundwater.
%%   \item As teaching assistance for this course, I have effectively
%%     communicated with both the lectures and students so that all the
%%     students can better apply the knowledge into the project using
%%     programming. 
%%   \item This course have significantly improved my skills on
%%     my teaching skill, in particular, how to deliver my knowledge to audiences so
%%     that they can understand them more intuitively.
%% \end{enumerate}
\end{document}
